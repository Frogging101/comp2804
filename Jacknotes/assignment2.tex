\documentclass[12pt]{article}

\usepackage{fullpage}
\usepackage{amssymb}
\usepackage{graphicx}
\usepackage{fancybox}

\newcommand{\IN}{\mathbb{N}}
\newcommand{\IZ}{\mathbb{Z}}

\newcommand{\silly}{\textsf{\sc Silly}}

\newcounter{ques}
\newenvironment{question}{\stepcounter{ques}{\noindent\bf Question \arabic{ques}:}}{\vspace{5mm}}

%%%%%%%%%%%%%% Capsule %%%%%%%%%%%%%%%%%%%%%%%%%%%%%%%%%%%%%%%%%%%
\newcommand{\capsule}[2]{\vspace{0.5em}
  \shadowbox{%
    \begin{minipage}{.90\linewidth}%
      \textbf{#1:}~#2%
    \end{minipage}}
  \vspace{0.5em} }
%%%%%%%%%%%%%%%%%%%%%%%%%%%%%%%%%%%%%%%%%%%%%%%%%%%%%%%%%%%%%%%%%%




\begin{document} 

\begin{center} \Large\bf
COMP 2804 --- Assignment 2 
\end{center} 

\noindent {\bf Due:} February 13, before 9:00 am, in the course drop 
box in Herzberg 3115. Note that 3115 is open from 8:30am until 4:30pm. 

\vspace{0.5em} 

\noindent {\bf Assignment Policy:} 
Late assignments will not be accepted. Students are encouraged to
collaborate on assignments, but at the level of discussion only.
When writing the solutions, they should do so in their own
words. Past experience has shown conclusively that those who do not
put adequate effort into the assignments do not learn the material
and have a probability near 1 of doing poorly on the exams.

\vspace{0.5em} 

\noindent {\bf Important note:} When writing your solutions, you
must follow the guidelines below.
\begin{itemize}
\item The answers should be concise, clear and neat.
\item When presenting proofs, every step should be justified.
\item Assignments should be stapled or placed in an unsealed envelope.
\end{itemize}
Substantial departures from the above guidelines will not be graded.

\vspace{1em} 

\begin{question} 
On the first page of your assignment, write your name and student number. 
\end{question} 

\begin{question} 
The functions $f : \IN \rightarrow \IN$ and $g : \IN \rightarrow \IN$ 
are recursively defined as follows:
\[ \begin{array}{lcll} 
      f(0) & = & 1 , & \\ 
      f(n) & = & g(n,f(n-1)) & \mbox{if $n \geq 1$,}  \\ 
      g(m,0) & = & 0 & \mbox{if $m \geq 0$,} \\ 
      g(m,n) & = & m + g(m,n-1) & \mbox{if $m \geq 0$ and $n \geq 1$.}
   \end{array} 
\] 
Solve these recurrences for $f$, i.e., express $f(n)$ in terms of $n$.  
Justify your answer. 
\end{question} 

\begin{question} 
The function $f : \IN \rightarrow \IZ$ is defined by 
\[ f(n) = 2n ( n-6) 
\]
for each integer $n \geq 0$. Derive a recursive form of this 
function $f$. Show your work. 
\end{question} 

\begin{question}  
The function $f : \IN^3 \rightarrow \IN$ is defined as follows:
\[ \begin{array}{ccll} 
      f(k,n,0) & = & k+n & \mbox{if $k \geq 0$ and $n \geq 0$,} \\
      f(k,0,1) & = & 0 & \mbox{if $k \geq 0$,} \\
      f(k,0,2) & = & 1 & \mbox{if $k \geq 0$,} \\
      f(k,0,i) & = & k & \mbox{if $k \geq 0$ and $i \geq 3$,} \\
      f(k,n,i) & = & f ( k , f(k,n-1,i) , i-1) & 
                \mbox{if $k \geq 0$, $i \geq 1$, and $n \geq 1$.} 
   \end{array} 
\] 
Determine $f(2,3,2)$. Show your work. 
\end{question} 

\begin{question}  
A \emph{maximal run of ones} in a bitstring is a maximal consecutive
substring of ones. For example, the bitstring $1100011110100111$
has four maximal runs of ones: $11$, $1111$, $1$, and $111$. These 
maximal runs have lengths $2$, $4$, $1$, and $3$, respectively. 

Let $n \geq 1$ be an integer and let $B_n$ be the number of bitstrings 
of length $n$ that do not contain any maximal run of ones of odd 
length; in other words, every maximal run of ones in these bitstrings 
has an even length. 
\begin{itemize} 
\item Determine $B_1$, $B_2$, and $B_3$. 
\item Determine the value of $B_n$, i.e., express $B_n$ in terms of 
      numbers that we have seen in class. Justify your answer. 
      \emph{Hint:} Derive a recurrence. 
\end{itemize}  
\end{question} 

\begin{question} 
Let $n \geq 1$ be an integer and let $S_n$ be the number of ways in which 
$n$ can be written as a sum of $1$s and $2$s; the order in which the 
$1$s and $2$s occur in the sum matters. For example, $S_3 = 3$, because 
\[ 3 = 1+1+1 = 1+2 = 2+1 . 
\] 
\begin{itemize} 
\item Determine $S_1$, $S_2$, and $S_4$. 
\item Determine the value of $S_n$, i.e., express $S_n$ in terms of 
      numbers that we have seen in class. Justify your answer. 
      \emph{Hint:} Derive a recurrence. 
\end{itemize}  
\end{question} 

\begin{question}   
Let $n$ be a positive integer and consider a $5 \times n$ board $B_n$
consisting of $5n$ cells, each one having sides of length one. The top
part of the figure below shows $B_{12}$.

\begin{center}
\includegraphics[scale=0.50]{figB12}
\end{center}

A \emph{brick} is a horizontal or vertical board consisting of 
$2 \times 3 = 6$ cells; see the bottom part of the figure above.
A \emph{tiling} of the board $B_n$ is a placement of bricks on the
board such that
\begin{itemize}
\item the bricks exactly cover $B_n$ and
\item no two bricks overlap.
\end{itemize}
The figure below shows a tiling of $B_{12}$.

\begin{center}
\includegraphics[scale=0.50]{figB12tiling}
\end{center}

Let $T_n$ be the number of different tilings of the board $B_n$. 
\begin{itemize} 
\item Let $n \geq 6$ be a multiple of $6$. Determine the value of $T_n$.
      Justify your answer. \emph{Hint:} Derive a recurrence.  
\item Let $n$ be a positive integer that is not a multiple of $6$. 
      Prove that $T_n = 0$. 
\end{itemize}  
\end{question} 

\begin{question} 
Consider the following recursive algorithm $\silly$, which takes as 
input an integer $n \geq 1$ which is a power of $2$:

\capsule{Algorithm $\silly(n)$}{
\begin{quote}
\begin{tabbing}
{\bf if} $n=1$ \\
{\bf then} drink one pint of beer \\
{\bf else} \= {\bf if} $n=2$ \\
           \> {\bf then} fart once \\
           \> {\bf else} \= fart once; \\
           \>            \> $\silly(n/2)$; \\
           \>            \> fart once \\
           \> {\bf endif} \\
{\bf endif}
\end{tabbing}
\end{quote}
} 

For $n$ a power of $2$, let $F(n)$ be the number of times you fart when
running algorithm $\silly(n)$. 
\begin{itemize}
\item Determine the value of $F(n)$ and prove that your answer is correct. 
      \emph{Hint:} Derive a recurrence. 
\end{itemize} 
\end{question} 

\begin{question}  
Let $m \geq 1$ and $n \geq 1$ be integers. Consider $m$ horizontal lines 
and $n$ non-horizontal lines such that  
\begin{itemize} 
\item no two of the non-horizontal lines are parallel, 
\item no three of the $m+n$ lines intersect in one single point. 
\end{itemize} 
These lines divide the plane into regions (some of which are bounded and 
some of which are unbounded). Denote the number of these regions by 
$R_{m,n}$. From the figure below, you can see that $R_{4,3} = 23$. 

\begin{center}
\includegraphics[scale=0.5]{regionshorizontal}
\end{center}

\begin{itemize} 
\item Derive a recurrence for the numbers $R_{m,n}$ and use it to prove 
      that 
      \[ R_{m,n} = 1 + m(n+1) + {{n+1} \choose 2} .  
      \] 
\end{itemize} 
\end{question} 

\end{document} 
